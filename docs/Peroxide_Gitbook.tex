\documentclass[]{book}
\usepackage{lmodern}
\usepackage{amssymb,amsmath}
\usepackage{ifxetex,ifluatex}
\usepackage{fixltx2e} % provides \textsubscript
\ifnum 0\ifxetex 1\fi\ifluatex 1\fi=0 % if pdftex
  \usepackage[T1]{fontenc}
  \usepackage[utf8]{inputenc}
\else % if luatex or xelatex
  \ifxetex
    \usepackage{mathspec}
  \else
    \usepackage{fontspec}
  \fi
  \defaultfontfeatures{Ligatures=TeX,Scale=MatchLowercase}
\fi
% use upquote if available, for straight quotes in verbatim environments
\IfFileExists{upquote.sty}{\usepackage{upquote}}{}
% use microtype if available
\IfFileExists{microtype.sty}{%
\usepackage{microtype}
\UseMicrotypeSet[protrusion]{basicmath} % disable protrusion for tt fonts
}{}
\usepackage[margin=1in]{geometry}
\usepackage{hyperref}
\hypersetup{unicode=true,
            pdftitle={Peroxide Guide},
            pdfauthor={Axect},
            pdfborder={0 0 0},
            breaklinks=true}
\urlstyle{same}  % don't use monospace font for urls
\usepackage{natbib}
\bibliographystyle{apalike}
\usepackage{color}
\usepackage{fancyvrb}
\newcommand{\VerbBar}{|}
\newcommand{\VERB}{\Verb[commandchars=\\\{\}]}
\DefineVerbatimEnvironment{Highlighting}{Verbatim}{commandchars=\\\{\}}
% Add ',fontsize=\small' for more characters per line
\usepackage{framed}
\definecolor{shadecolor}{RGB}{248,248,248}
\newenvironment{Shaded}{\begin{snugshade}}{\end{snugshade}}
\newcommand{\AlertTok}[1]{\textcolor[rgb]{0.94,0.16,0.16}{#1}}
\newcommand{\AnnotationTok}[1]{\textcolor[rgb]{0.56,0.35,0.01}{\textbf{\textit{#1}}}}
\newcommand{\AttributeTok}[1]{\textcolor[rgb]{0.77,0.63,0.00}{#1}}
\newcommand{\BaseNTok}[1]{\textcolor[rgb]{0.00,0.00,0.81}{#1}}
\newcommand{\BuiltInTok}[1]{#1}
\newcommand{\CharTok}[1]{\textcolor[rgb]{0.31,0.60,0.02}{#1}}
\newcommand{\CommentTok}[1]{\textcolor[rgb]{0.56,0.35,0.01}{\textit{#1}}}
\newcommand{\CommentVarTok}[1]{\textcolor[rgb]{0.56,0.35,0.01}{\textbf{\textit{#1}}}}
\newcommand{\ConstantTok}[1]{\textcolor[rgb]{0.00,0.00,0.00}{#1}}
\newcommand{\ControlFlowTok}[1]{\textcolor[rgb]{0.13,0.29,0.53}{\textbf{#1}}}
\newcommand{\DataTypeTok}[1]{\textcolor[rgb]{0.13,0.29,0.53}{#1}}
\newcommand{\DecValTok}[1]{\textcolor[rgb]{0.00,0.00,0.81}{#1}}
\newcommand{\DocumentationTok}[1]{\textcolor[rgb]{0.56,0.35,0.01}{\textbf{\textit{#1}}}}
\newcommand{\ErrorTok}[1]{\textcolor[rgb]{0.64,0.00,0.00}{\textbf{#1}}}
\newcommand{\ExtensionTok}[1]{#1}
\newcommand{\FloatTok}[1]{\textcolor[rgb]{0.00,0.00,0.81}{#1}}
\newcommand{\FunctionTok}[1]{\textcolor[rgb]{0.00,0.00,0.00}{#1}}
\newcommand{\ImportTok}[1]{#1}
\newcommand{\InformationTok}[1]{\textcolor[rgb]{0.56,0.35,0.01}{\textbf{\textit{#1}}}}
\newcommand{\KeywordTok}[1]{\textcolor[rgb]{0.13,0.29,0.53}{\textbf{#1}}}
\newcommand{\NormalTok}[1]{#1}
\newcommand{\OperatorTok}[1]{\textcolor[rgb]{0.81,0.36,0.00}{\textbf{#1}}}
\newcommand{\OtherTok}[1]{\textcolor[rgb]{0.56,0.35,0.01}{#1}}
\newcommand{\PreprocessorTok}[1]{\textcolor[rgb]{0.56,0.35,0.01}{\textit{#1}}}
\newcommand{\RegionMarkerTok}[1]{#1}
\newcommand{\SpecialCharTok}[1]{\textcolor[rgb]{0.00,0.00,0.00}{#1}}
\newcommand{\SpecialStringTok}[1]{\textcolor[rgb]{0.31,0.60,0.02}{#1}}
\newcommand{\StringTok}[1]{\textcolor[rgb]{0.31,0.60,0.02}{#1}}
\newcommand{\VariableTok}[1]{\textcolor[rgb]{0.00,0.00,0.00}{#1}}
\newcommand{\VerbatimStringTok}[1]{\textcolor[rgb]{0.31,0.60,0.02}{#1}}
\newcommand{\WarningTok}[1]{\textcolor[rgb]{0.56,0.35,0.01}{\textbf{\textit{#1}}}}
\usepackage{longtable,booktabs}
\usepackage{graphicx,grffile}
\makeatletter
\def\maxwidth{\ifdim\Gin@nat@width>\linewidth\linewidth\else\Gin@nat@width\fi}
\def\maxheight{\ifdim\Gin@nat@height>\textheight\textheight\else\Gin@nat@height\fi}
\makeatother
% Scale images if necessary, so that they will not overflow the page
% margins by default, and it is still possible to overwrite the defaults
% using explicit options in \includegraphics[width, height, ...]{}
\setkeys{Gin}{width=\maxwidth,height=\maxheight,keepaspectratio}
\IfFileExists{parskip.sty}{%
\usepackage{parskip}
}{% else
\setlength{\parindent}{0pt}
\setlength{\parskip}{6pt plus 2pt minus 1pt}
}
\setlength{\emergencystretch}{3em}  % prevent overfull lines
\providecommand{\tightlist}{%
  \setlength{\itemsep}{0pt}\setlength{\parskip}{0pt}}
\setcounter{secnumdepth}{5}
% Redefines (sub)paragraphs to behave more like sections
\ifx\paragraph\undefined\else
\let\oldparagraph\paragraph
\renewcommand{\paragraph}[1]{\oldparagraph{#1}\mbox{}}
\fi
\ifx\subparagraph\undefined\else
\let\oldsubparagraph\subparagraph
\renewcommand{\subparagraph}[1]{\oldsubparagraph{#1}\mbox{}}
\fi

%%% Use protect on footnotes to avoid problems with footnotes in titles
\let\rmarkdownfootnote\footnote%
\def\footnote{\protect\rmarkdownfootnote}

%%% Change title format to be more compact
\usepackage{titling}

% Create subtitle command for use in maketitle
\newcommand{\subtitle}[1]{
  \posttitle{
    \begin{center}\large#1\end{center}
    }
}

\setlength{\droptitle}{-2em}

  \title{Peroxide Guide}
    \pretitle{\vspace{\droptitle}\centering\huge}
  \posttitle{\par}
    \author{Axect}
    \preauthor{\centering\large\emph}
  \postauthor{\par}
      \predate{\centering\large\emph}
  \postdate{\par}
    \date{2019-03-07}

\usepackage{booktabs}
\usepackage{amsthm}
\makeatletter
\def\thm@space@setup{%
  \thm@preskip=8pt plus 2pt minus 4pt
  \thm@postskip=\thm@preskip
}
\makeatother

\begin{document}
\maketitle

{
\setcounter{tocdepth}{1}
\tableofcontents
}
\hypertarget{prerequisites}{%
\chapter{Prerequisites}\label{prerequisites}}

\begin{itemize}
\tightlist
\item
  \href{https://www.rust-lang.org/tools/install}{Rust}
\item
  \href{https://crates.io/crates/peroxide}{Peroxide}
\end{itemize}

\hypertarget{quick}{%
\chapter{Quick Start}\label{quick}}

\hypertarget{cargo.toml}{%
\section{Cargo.toml}\label{cargo.toml}}

\begin{itemize}
\item
  To use \texttt{peroxide}, you should edit \texttt{Cargo.toml}
\item
  Current document version is corresponding to \texttt{0.8}

\begin{verbatim}
peroxide = "0.8"
\end{verbatim}
\end{itemize}

\hypertarget{import-all-at-once}{%
\section{Import all at once}\label{import-all-at-once}}

\begin{itemize}
\item
  You can import all functions \& structures at once

\begin{Shaded}
\begin{Highlighting}[]
\KeywordTok{extern} \KeywordTok{crate}\NormalTok{ peroxide;}
\KeywordTok{use} \PreprocessorTok{peroxide::}\NormalTok{*;}

\KeywordTok{fn}\NormalTok{ main() }\OperatorTok{\{}
    \CommentTok{//Some codes...}
\OperatorTok{\}}
\end{Highlighting}
\end{Shaded}
\end{itemize}

\hypertarget{vector}{%
\chapter{Vector}\label{vector}}

\hypertarget{print-vecf64}{%
\section{\texorpdfstring{Print \texttt{Vec\textless{}f64\textgreater{}}}{Print Vec\textless{}f64\textgreater{}}}\label{print-vecf64}}

\begin{itemize}
\tightlist
\item
  There are two ways to print vector

  \begin{itemize}
  \tightlist
  \item
    Original way: \texttt{print!("\{:?\}",\ a);}
  \item
    Peroxide way: \texttt{a.print();} - \textbf{Round-off to fourth digit}
  \end{itemize}

\begin{Shaded}
\begin{Highlighting}[]
\KeywordTok{fn}\NormalTok{ main() }\OperatorTok{\{}
    \KeywordTok{let}\NormalTok{ a = }\PreprocessorTok{vec!}\OperatorTok{[}\DecValTok{2f64}\NormalTok{.sqrt()}\OperatorTok{]}\NormalTok{;}
\NormalTok{    a.print(); }\CommentTok{// [1.4142]}
\OperatorTok{\}}
\end{Highlighting}
\end{Shaded}
\end{itemize}

\hypertarget{syntactic-sugar-for-vecf64}{%
\section{\texorpdfstring{Syntactic sugar for \texttt{Vec\textless{}f64\textgreater{}}}{Syntactic sugar for Vec\textless{}f64\textgreater{}}}\label{syntactic-sugar-for-vecf64}}

\begin{itemize}
\item
  There is useful macro for \texttt{Vec\textless{}f64\textgreater{}}
\item
  For \texttt{R}, there is \texttt{c}

\begin{Shaded}
\begin{Highlighting}[]
\CommentTok{# R}
\NormalTok{a =}\StringTok{ }\KeywordTok{c}\NormalTok{(}\DecValTok{1}\NormalTok{,}\DecValTok{2}\NormalTok{,}\DecValTok{3}\NormalTok{,}\DecValTok{4}\NormalTok{)}
\end{Highlighting}
\end{Shaded}
\item
  For \texttt{Peroxide}, there is \texttt{c!}

\begin{Shaded}
\begin{Highlighting}[]
\CommentTok{// Rust}
\KeywordTok{fn}\NormalTok{ main() }\OperatorTok{\{}
    \KeywordTok{let}\NormalTok{ a = }\PreprocessorTok{c!}\NormalTok{(}\DecValTok{1}\NormalTok{,}\DecValTok{2}\NormalTok{,}\DecValTok{3}\NormalTok{,}\DecValTok{4}\NormalTok{);}
\OperatorTok{\}}
\end{Highlighting}
\end{Shaded}
\end{itemize}

\hypertarget{from-ranges-to-vector}{%
\section{From ranges to Vector}\label{from-ranges-to-vector}}

\begin{itemize}
\item
  For \texttt{R}, there is \texttt{seq} to declare sequence.

\begin{Shaded}
\begin{Highlighting}[]
\CommentTok{# R}
\NormalTok{a =}\StringTok{ }\KeywordTok{seq}\NormalTok{(}\DecValTok{1}\NormalTok{, }\DecValTok{4}\NormalTok{, }\DecValTok{1}\NormalTok{)}
\KeywordTok{print}\NormalTok{(a)}
\CommentTok{# [1] 1 2 3 4}
\end{Highlighting}
\end{Shaded}
\item
  For \texttt{peroxide}, there is \texttt{seq} to declare sequence.

\begin{Shaded}
\begin{Highlighting}[]
\KeywordTok{fn}\NormalTok{ main() }\OperatorTok{\{}
    \KeywordTok{let}\NormalTok{ a = seq(}\DecValTok{1}\NormalTok{, }\DecValTok{4}\NormalTok{, }\DecValTok{1}\NormalTok{);}
\NormalTok{    a.print();}
    \CommentTok{// [1, 2, 3, 4]}
\OperatorTok{\}}
\end{Highlighting}
\end{Shaded}
\end{itemize}

\hypertarget{vector-operation}{%
\section{Vector Operation}\label{vector-operation}}

\begin{itemize}
\tightlist
\item
  There are some vector-wise operations

  \begin{itemize}
  \tightlist
  \item
    \texttt{add(\&self,\ other:\ Vec\textless{}f64\textgreater{})\ -\textgreater{}\ Vec\textless{}f64\textgreater{}}
  \item
    \texttt{sub(\&self,\ other:\ Vec\textless{}f64\textgreater{})\ -\textgreater{}\ Vec\textless{}f64\textgreater{}}
  \item
    \texttt{mul(\&self,\ other:\ Vec\textless{}f64\textgreater{})\ -\textgreater{}\ Vec\textless{}f64\textgreater{}}
  \item
    \texttt{div(\&self,\ other:\ Vec\textless{}f64\textgreater{})\ -\textgreater{}\ Vec\textless{}f64\textgreater{}}
  \item
    \texttt{dot(\&self,\ other:\ Vec\textless{}f64\textgreater{})\ -\textgreater{}\ f64}
  \item
    \texttt{norm(\&self)\ -\textgreater{}\ f64}
  \end{itemize}

\begin{Shaded}
\begin{Highlighting}[]
\KeywordTok{fn}\NormalTok{ main() }\OperatorTok{\{}
    \KeywordTok{let}\NormalTok{ a = }\PreprocessorTok{c!}\NormalTok{(}\DecValTok{1}\NormalTok{,}\DecValTok{2}\NormalTok{,}\DecValTok{3}\NormalTok{,}\DecValTok{4}\NormalTok{);}
    \KeywordTok{let}\NormalTok{ b = }\PreprocessorTok{c!}\NormalTok{(}\DecValTok{4}\NormalTok{,}\DecValTok{3}\NormalTok{,}\DecValTok{2}\NormalTok{,}\DecValTok{1}\NormalTok{);}

\NormalTok{    a.add(&b).print();}
\NormalTok{    a.sub(&b).print();}
\NormalTok{    a.mul(&b).print();}
\NormalTok{    a.div(&b).print();}
\NormalTok{    a.dot(&b).print();}
\NormalTok{    a.norm().print();}

    \CommentTok{// [5, 5, 5, 5]}
    \CommentTok{// [-3, -1, 1, 3]}
    \CommentTok{// [4, 6, 6, 4]}
    \CommentTok{// [0.25, 0.6667, 1.5, 4]}
    \CommentTok{// 20}
    \CommentTok{// 5.477225575051661 // sqrt(30)}
\OperatorTok{\}}
\end{Highlighting}
\end{Shaded}
\item
  And there are some useful operations too.

  \begin{itemize}
  \tightlist
  \item
    \texttt{pow(\&self,\ usize)\ -\textgreater{}\ Vec\textless{}f64\textgreater{}}
  \item
    \texttt{powf(\&self,\ f64)\ -\textgreater{}\ Vec\textless{}f64\textgreater{}}
  \item
    \texttt{sqrt(\&self)\ -\textgreater{}\ Vec\textless{}f64\textgreater{}}
  \end{itemize}

\begin{Shaded}
\begin{Highlighting}[]
\KeywordTok{fn}\NormalTok{ main() }\OperatorTok{\{}
    \KeywordTok{let}\NormalTok{ a = }\PreprocessorTok{c!}\NormalTok{(}\DecValTok{1}\NormalTok{,}\DecValTok{2}\NormalTok{,}\DecValTok{3}\NormalTok{,}\DecValTok{4}\NormalTok{);}

\NormalTok{    a.pow(}\DecValTok{2}\NormalTok{).print();}
\NormalTok{    a.powf(}\DecValTok{0.5}\NormalTok{).print();}
\NormalTok{    a.sqrt().print();}
    \CommentTok{// [1, 4, 9, 16]}
    \CommentTok{// [1, 1.4142, 1.7321, 2]}
    \CommentTok{// [1, 1.4142, 1.7321, 2]}
\OperatorTok{\}}
\end{Highlighting}
\end{Shaded}
\end{itemize}

\hypertarget{concatenation}{%
\section{Concatenation}\label{concatenation}}

There are two concatenation operations.

\begin{itemize}
\item
  \texttt{cat(T,\ Vec\textless{}T\textgreater{})\ -\textgreater{}\ Vec\textless{}f64\textgreater{}}
\item
  \texttt{concat(Vec\textless{}T\textgreater{},\ Vec\textless{}T\textgreater{})\ -\textgreater{}\ Vec\textless{}T\textgreater{}}

\begin{Shaded}
\begin{Highlighting}[]
\KeywordTok{fn}\NormalTok{ main() }\OperatorTok{\{}
    \KeywordTok{let}\NormalTok{ a = }\PreprocessorTok{c!}\NormalTok{(}\DecValTok{1}\NormalTok{,}\DecValTok{2}\NormalTok{,}\DecValTok{3}\NormalTok{,}\DecValTok{4}\NormalTok{);}
\NormalTok{    cat(}\DecValTok{0f64}\NormalTok{, a.clone()).print();}
    \CommentTok{// [0, 1, 2, 3, 4]}

    \KeywordTok{let}\NormalTok{ b = }\PreprocessorTok{c!}\NormalTok{(}\DecValTok{5}\NormalTok{,}\DecValTok{6}\NormalTok{,}\DecValTok{7}\NormalTok{,}\DecValTok{8}\NormalTok{);}
\NormalTok{    concat(a, b).print();}
    \CommentTok{// [1, 2, 3, 4, 5, 6, 7, 8]}
\OperatorTok{\}}
\end{Highlighting}
\end{Shaded}
\end{itemize}

\hypertarget{matrix}{%
\chapter{Matrix}\label{matrix}}

\hypertarget{declare-matrix}{%
\section{Declare matrix}\label{declare-matrix}}

\begin{itemize}
\tightlist
\item
  You can declare matrix by various ways.

  \begin{itemize}
  \tightlist
  \item
    R's way - Default
  \item
    MATLAB's way
  \item
    Python's way
  \item
    Other macro
  \end{itemize}
\end{itemize}

\hypertarget{rs-way}{%
\subsection{R's way}\label{rs-way}}

\begin{itemize}
\tightlist
\item
  Description: Same as R - \texttt{matrix(Vector,\ Row,\ Col,\ Shape)}
\item
  Type: \texttt{matrix(Vec\textless{}T\textgreater{},\ usize,\ usize,\ Shape)\ where\ T:\ std::convert::Into\textless{}f64\textgreater{}\ +\ Copy}

  \begin{itemize}
  \tightlist
  \item
    \texttt{Shape}: \texttt{Enum} for matrix shape - \texttt{Row} \& \texttt{Col}
  \end{itemize}

\begin{Shaded}
\begin{Highlighting}[]
\KeywordTok{fn}\NormalTok{ main() }\OperatorTok{\{}
    \KeywordTok{let}\NormalTok{ a = matrix(}\PreprocessorTok{c!}\NormalTok{(}\DecValTok{1}\NormalTok{,}\DecValTok{2}\NormalTok{,}\DecValTok{3}\NormalTok{,}\DecValTok{4}\NormalTok{), }\DecValTok{2}\NormalTok{, }\DecValTok{2}\NormalTok{, Row);}
\NormalTok{    a.print();}
    \CommentTok{//       c[0] c[1]}
    \CommentTok{// r[0]     1    2}
    \CommentTok{// r[1]     3    4}

    \KeywordTok{let}\NormalTok{ b = matrix(}\PreprocessorTok{c!}\NormalTok{(}\DecValTok{1}\NormalTok{,}\DecValTok{2}\NormalTok{,}\DecValTok{3}\NormalTok{,}\DecValTok{4}\NormalTok{), }\DecValTok{2}\NormalTok{, }\DecValTok{2}\NormalTok{, Col);}
\NormalTok{    b.print();}
    \CommentTok{//       c[0] c[1]}
    \CommentTok{// r[0]     1    3}
    \CommentTok{// r[1]     2    4}
\OperatorTok{\}}
\end{Highlighting}
\end{Shaded}
\end{itemize}

\hypertarget{matlabs-way}{%
\subsection{MATLAB's way}\label{matlabs-way}}

\begin{itemize}
\item
  Description: Similar to MATLAB (But should use \texttt{\&str})
\item
  Type: \texttt{ml\_matrix(\&str)}

\begin{Shaded}
\begin{Highlighting}[]
\KeywordTok{fn}\NormalTok{ main() }\OperatorTok{\{}
    \KeywordTok{let}\NormalTok{ a = ml_matrix(}\StringTok{"1 2; 3 4"}\NormalTok{);}
\NormalTok{    a.print();}
    \CommentTok{//       c[0] c[1]}
    \CommentTok{// r[0]     1    2}
    \CommentTok{// r[1]     3    4}
\OperatorTok{\}}
\end{Highlighting}
\end{Shaded}
\end{itemize}

\hypertarget{pythons-way}{%
\subsection{Python's way}\label{pythons-way}}

\begin{itemize}
\item
  Description: Declare matrix as vector of vectors.
\item
  Type: \texttt{py\_matrix(Vec\textless{}Vec\textless{}T\textgreater{}\textgreater{})\ where\ T:\ std::convert::Into\textless{}f64\textgreater{}\ +\ Copy}

\begin{Shaded}
\begin{Highlighting}[]
\KeywordTok{fn}\NormalTok{ main() }\OperatorTok{\{}
    \KeywordTok{let}\NormalTok{ a = py_matrix(}\PreprocessorTok{vec!}\OperatorTok{[}\PreprocessorTok{vec!}\OperatorTok{[}\DecValTok{1}\NormalTok{, }\DecValTok{2}\OperatorTok{]}\NormalTok{, }\PreprocessorTok{vec!}\OperatorTok{[}\DecValTok{3}\NormalTok{, }\DecValTok{4}\OperatorTok{]]}\NormalTok{);}
\NormalTok{    a.print();}
    \CommentTok{//       c[0] c[1]}
    \CommentTok{// r[0]     1    2}
    \CommentTok{// r[1]     3    4}
\OperatorTok{\}}
\end{Highlighting}
\end{Shaded}
\end{itemize}

\hypertarget{other-macro}{%
\subsection{Other macro}\label{other-macro}}

\begin{itemize}
\item
  Description: R-like macro to declare matrix
\item
  For \texttt{R},

\begin{Shaded}
\begin{Highlighting}[]
\CommentTok{# R}
\NormalTok{a =}\StringTok{ }\KeywordTok{matrix}\NormalTok{(}\DecValTok{1}\OperatorTok{:}\DecValTok{4}\OperatorTok{:}\DecValTok{1}\NormalTok{, }\DecValTok{2}\NormalTok{, }\DecValTok{2}\NormalTok{, Row)}
\KeywordTok{print}\NormalTok{(a)}
\CommentTok{#      [,1] [,2]}
\CommentTok{# [1,]    1    2}
\CommentTok{# [2,]    3    4}
\end{Highlighting}
\end{Shaded}
\item
  For \texttt{Peroxide},

\begin{Shaded}
\begin{Highlighting}[]
\KeywordTok{fn}\NormalTok{ main() }\OperatorTok{\{}
    \KeywordTok{let}\NormalTok{ a = }\PreprocessorTok{matrix!}\NormalTok{(}\DecValTok{1}\NormalTok{;}\DecValTok{4}\NormalTok{;}\DecValTok{1}\NormalTok{, }\DecValTok{2}\NormalTok{, }\DecValTok{2}\NormalTok{, Row);}
\NormalTok{    a.print();}
    \CommentTok{//       c[0] c[1]}
    \CommentTok{// r[0]     1    2}
    \CommentTok{// r[1]     3    4}
\OperatorTok{\}}
\end{Highlighting}
\end{Shaded}
\end{itemize}

\hypertarget{basic-method-for-matrix}{%
\section{Basic Method for Matrix}\label{basic-method-for-matrix}}

There are some useful methods for \texttt{Matrix}

\begin{itemize}
\item
  \texttt{row(\&self,\ index:\ usize)\ -\textgreater{}\ Vec\textless{}f64\textgreater{}} : Extract specific row as \texttt{Vec\textless{}f64\textgreater{}}
\item
  \texttt{col(\&self,\ index:\ usize)\ -\textgreater{}\ Vec\textless{}f64\textgreater{}} : Extract specific column as \texttt{Vec\textless{}f64\textgreater{}}
\item
  \texttt{diag(\&self)\ -\textgreater{}\ Vec\textless{}f64\textgreater{}}: Extract diagonal components as \texttt{Vec\textless{}f64\textgreater{}}
\item
  \texttt{swap(\&self,\ usize,\ usize,\ Shape)\ -\textgreater{}\ Matrix}: Swap two rows or columns
\item
  \texttt{subs\_col(\&mut\ self,\ usize,\ Vec\textless{}f64\textgreater{})}: Substitute column with \texttt{Vec\textless{}f64\textgreater{}}
\item
  \texttt{subs\_row(\&mut\ self,\ usize,\ Vec\textless{}f64\textgreater{})}: Substitute row with \texttt{Vec\textless{}f64\textgreater{}}

\begin{Shaded}
\begin{Highlighting}[]
\KeywordTok{fn}\NormalTok{ main() }\OperatorTok{\{}
    \KeywordTok{let}\NormalTok{ a = ml_matrix(}\StringTok{"1 2; 3 4"}\NormalTok{);}

\NormalTok{    a.row(}\DecValTok{0}\NormalTok{).print(); }\CommentTok{// [1, 2]}
\NormalTok{    a.col(}\DecValTok{0}\NormalTok{).print(); }\CommentTok{// [1, 3]}
\NormalTok{    a.diag().print(); }\CommentTok{// [1, 4]}
\NormalTok{    a.swap(}\DecValTok{0}\NormalTok{, }\DecValTok{1}\NormalTok{, Row).print();}
    \CommentTok{//      c[0] c[1]}
    \CommentTok{// r[0]    3    4}
    \CommentTok{// r[1]    1    2}

    \KeywordTok{let} \KeywordTok{mut}\NormalTok{ b = ml_matrix(}\StringTok{"1 2;3 4"}\NormalTok{);}
\NormalTok{    b.subs_col(}\DecValTok{0}\NormalTok{, }\PreprocessorTok{c!}\NormalTok{(}\DecValTok{5}\NormalTok{, }\DecValTok{6}\NormalTok{));}
\NormalTok{    b.subs_row(}\DecValTok{1}\NormalTok{, }\PreprocessorTok{c!}\NormalTok{(}\DecValTok{7}\NormalTok{, }\DecValTok{8}\NormalTok{));}
\NormalTok{    b.print();}
    \CommentTok{//       c[0] c[1]}
    \CommentTok{// r[0]    5    2}
    \CommentTok{// r[1]    7    8}
\OperatorTok{\}}
\end{Highlighting}
\end{Shaded}
\end{itemize}

\hypertarget{read-write}{%
\section{Read \& Write}\label{read-write}}

In peroxide, we can write matrix to csv.

\begin{itemize}
\item
  \texttt{write(\&self,\ file\_path:\ \&str)}: Write matrix to csv
\item
  \texttt{write\_with\_header(\&self,\ file\_path,\ header:\ Vec\textless{}\&str\textgreater{})}: Write with header

\begin{Shaded}
\begin{Highlighting}[]
\KeywordTok{fn}\NormalTok{ main() }\OperatorTok{\{}
    \KeywordTok{let}\NormalTok{ a = ml_matrix(}\StringTok{"1 2;3 4"}\NormalTok{);}
\NormalTok{    a.write(}\StringTok{"matrix.csv"}\NormalTok{).expect(}\StringTok{"Can't write file"}\NormalTok{);}

    \KeywordTok{let}\NormalTok{ b = ml_matrix(}\StringTok{"1 2; 3 4; 5 6"}\NormalTok{);}
\NormalTok{    b.write_with_header(}\StringTok{"header.csv"}\NormalTok{, }\PreprocessorTok{vec!}\OperatorTok{[}\StringTok{"odd"}\NormalTok{, }\StringTok{"even"}\OperatorTok{]}\NormalTok{)}
\NormalTok{        .expect(}\StringTok{"Can't write header file"}\NormalTok{);}
\OperatorTok{\}}
\end{Highlighting}
\end{Shaded}
\end{itemize}

Also, you can read matrix from csv.

\begin{itemize}
\item
  Type: \texttt{read(\&str,\ bool,\ char)\ -\textgreater{}\ Result\textless{}Matrix,\ Box\textless{}Error\textgreater{}\textgreater{}}
\item
  Description: \texttt{read(file\_path,\ is\_header,\ delimiter)}

\begin{Shaded}
\begin{Highlighting}[]
\KeywordTok{fn}\NormalTok{ main() }\OperatorTok{\{}
    \KeywordTok{let}\NormalTok{ a = read(}\StringTok{"matrix.csv"}\NormalTok{, }\ConstantTok{false}\NormalTok{, }\CharTok{','}\NormalTok{)}
\NormalTok{        .expect(}\StringTok{"Can't read matrix.csv file"}\NormalTok{);}
\NormalTok{    a.print();}
    \CommentTok{//       c[0] c[1]}
    \CommentTok{// r[0]     1    2}
    \CommentTok{// r[1]     3    4}
\OperatorTok{\}}
\end{Highlighting}
\end{Shaded}
\end{itemize}

\hypertarget{concatenation-1}{%
\section{Concatenation}\label{concatenation-1}}

There are two options to concatenate matrices.

\begin{itemize}
\item
  \texttt{cbind}: Concatenate two matrices by column direction.
\item
  \texttt{rbind}: Concatenate two matrices by row direction.

\begin{Shaded}
\begin{Highlighting}[]
\KeywordTok{fn}\NormalTok{ main() }\OperatorTok{\{}
    \KeywordTok{let}\NormalTok{ a = ml_matrix(}\StringTok{"1 2;3 4"}\NormalTok{);}
    \KeywordTok{let}\NormalTok{ b = ml_matrix(}\StringTok{"5 6;7 8"}\NormalTok{);}

\NormalTok{    cbind(a.clone(), b.clone()).print();}
    \CommentTok{//      c[0] c[1] c[2] c[3]}
    \CommentTok{// r[0]    1    2    5    7}
    \CommentTok{// r[1]    3    4    6    8}

\NormalTok{    rbind(a, b).print();}
    \CommentTok{//      c[0] c[1]}
    \CommentTok{// r[0]    1    2}
    \CommentTok{// r[1]    3    4}
    \CommentTok{// r[2]    5    6}
    \CommentTok{// r[3]    7    8}
\OperatorTok{\}}
\end{Highlighting}
\end{Shaded}
\end{itemize}

\hypertarget{matrix-operations}{%
\section{Matrix operations}\label{matrix-operations}}

\begin{itemize}
\item
  In peroxide, can use basic operations between matrices. I'll show you by examples.

\begin{Shaded}
\begin{Highlighting}[]
\KeywordTok{fn}\NormalTok{ main() }\OperatorTok{\{}
    \KeywordTok{let}\NormalTok{ a = }\PreprocessorTok{matrix!}\NormalTok{(}\DecValTok{1}\NormalTok{;}\DecValTok{4}\NormalTok{;}\DecValTok{1}\NormalTok{, }\DecValTok{2}\NormalTok{, }\DecValTok{2}\NormalTok{, Row);}
\NormalTok{    (a.clone() + }\DecValTok{1}\NormalTok{).print(); }\CommentTok{// -, *, / are also available}
    \CommentTok{//      c[0] c[1]}
    \CommentTok{// r[0]    2    3}
    \CommentTok{// r[1]    4    5}

    \KeywordTok{let}\NormalTok{ b = }\PreprocessorTok{matrix!}\NormalTok{(}\DecValTok{5}\NormalTok{;}\DecValTok{8}\NormalTok{;}\DecValTok{1}\NormalTok{, }\DecValTok{2}\NormalTok{, }\DecValTok{2}\NormalTok{, Row);}
\NormalTok{    (a.clone() + b.clone()).print(); }\CommentTok{// -, *, / are also available}
    \CommentTok{//      c[0] c[1]}
    \CommentTok{// r[0]    6    8}
    \CommentTok{// r[1]   10   12}

\NormalTok{    (a.clone() % b.clone()).print(); }\CommentTok{// Matrix multiplication}
    \CommentTok{//      c[0] c[1]}
    \CommentTok{// r[0]   19   22}
    \CommentTok{// r[1]   43   50}
\OperatorTok{\}}
\end{Highlighting}
\end{Shaded}
\end{itemize}

\hypertarget{extract-modify-components}{%
\section{Extract \& modify components}\label{extract-modify-components}}

\begin{itemize}
\item
  In peroxide, matrix data is saved as linear structure.
\item
  But you can use two-dimensional index to extract or modify components.

\begin{Shaded}
\begin{Highlighting}[]
\KeywordTok{fn}\NormalTok{ main() }\OperatorTok{\{}
    \KeywordTok{let} \KeywordTok{mut}\NormalTok{ a = }\PreprocessorTok{matrix!}\NormalTok{(}\DecValTok{1}\NormalTok{;}\DecValTok{4}\NormalTok{;}\DecValTok{1}\NormalTok{, }\DecValTok{2}\NormalTok{, }\DecValTok{2}\NormalTok{, Row);}
\NormalTok{    a}\OperatorTok{[}\NormalTok{(}\DecValTok{0}\NormalTok{,}\DecValTok{0}\NormalTok{)}\OperatorTok{]}\NormalTok{.print(); }\CommentTok{// 1}
\NormalTok{    a}\OperatorTok{[}\NormalTok{(}\DecValTok{0}\NormalTok{,}\DecValTok{0}\NormalTok{)}\OperatorTok{]}\NormalTok{ = }\DecValTok{2f64}\NormalTok{; }\CommentTok{// Modify component}
\NormalTok{    a.print();}
    \CommentTok{//       c[0] c[1]}
    \CommentTok{//  r[0]    2    2}
    \CommentTok{//  r[1]    3    4}
\OperatorTok{\}}
\end{Highlighting}
\end{Shaded}
\end{itemize}

\hypertarget{conversion-between-vector}{%
\section{Conversion between vector}\label{conversion-between-vector}}

\hypertarget{vector-to-matrix}{%
\subsection{Vector to Matrix}\label{vector-to-matrix}}

\begin{itemize}
\item
  \texttt{to\_matrix} method allows conversion from vector to column matrix.

\begin{Shaded}
\begin{Highlighting}[]
\KeywordTok{fn}\NormalTok{ main() }\OperatorTok{\{}
    \KeywordTok{let}\NormalTok{ a = }\PreprocessorTok{c!}\NormalTok{(}\DecValTok{1}\NormalTok{,}\DecValTok{2}\NormalTok{,}\DecValTok{3}\NormalTok{,}\DecValTok{4}\NormalTok{);}
\NormalTok{    a.to_matrix().print();}
    \CommentTok{//      c[0]}
    \CommentTok{// r[0]    1}
    \CommentTok{// r[1]    2}
    \CommentTok{// r[2]    3}
    \CommentTok{// r[3]    4}
\OperatorTok{\}}
\end{Highlighting}
\end{Shaded}
\end{itemize}

\hypertarget{matrix-to-vector}{%
\subsection{Matrix to Vector}\label{matrix-to-vector}}

\begin{itemize}
\item
  Just use \texttt{row} or \texttt{col} method (I already showed at Basic method section).

\begin{Shaded}
\begin{Highlighting}[]
\KeywordTok{fn}\NormalTok{ main() }\OperatorTok{\{}
    \KeywordTok{let}\NormalTok{ a = }\PreprocessorTok{matrix!}\NormalTok{(}\DecValTok{1}\NormalTok{;}\DecValTok{4}\NormalTok{;}\DecValTok{1}\NormalTok{, }\DecValTok{2}\NormalTok{, }\DecValTok{2}\NormalTok{, Row);}
\NormalTok{    a.row(}\DecValTok{0}\NormalTok{).print(); }\CommentTok{// [1, 2]}
\OperatorTok{\}}
\end{Highlighting}
\end{Shaded}
\end{itemize}

\hypertarget{useful-constructor}{%
\section{Useful constructor}\label{useful-constructor}}

\begin{itemize}
\item
  \texttt{zeros(usize,\ usize)}: Construct matrix which elements are all zero
\item
  \texttt{eye(usize)}: Identity matrix
\item
  \texttt{rand(usize,\ usize)}: Construct random uniform matrix (from 0 to 1)

\begin{Shaded}
\begin{Highlighting}[]
\KeywordTok{fn}\NormalTok{ main() }\OperatorTok{\{}
    \KeywordTok{let}\NormalTok{ a = zeros(}\DecValTok{2}\NormalTok{, }\DecValTok{2}\NormalTok{);}
    \PreprocessorTok{assert_eq!}\NormalTok{(a, ml_matrix(}\StringTok{"0 0;0 0"}\NormalTok{));}

    \KeywordTok{let}\NormalTok{ b = eye(}\DecValTok{2}\NormalTok{);}
    \PreprocessorTok{assert_eq!}\NormalTok{(b, ml_matrix(}\StringTok{"1 0;0 1"}\NormalTok{));}

    \KeywordTok{let}\NormalTok{ c = rand(}\DecValTok{2}\NormalTok{, }\DecValTok{2}\NormalTok{);}
\NormalTok{    c.print(); }\CommentTok{// Random 2x2 matrix}
\OperatorTok{\}}
\end{Highlighting}
\end{Shaded}
\end{itemize}

\hypertarget{linear}{%
\chapter{Linear Algebra}\label{linear}}

\hypertarget{transpose}{%
\section{Transpose}\label{transpose}}

\begin{itemize}
\item
  Caution: Transpose does not consume the original value.

\begin{Shaded}
\begin{Highlighting}[]
\KeywordTok{fn}\NormalTok{ main() }\OperatorTok{\{}
    \KeywordTok{let}\NormalTok{ a = }\PreprocessorTok{matrix!}\NormalTok{(}\DecValTok{1}\NormalTok{;}\DecValTok{4}\NormalTok{;}\DecValTok{1}\NormalTok{, }\DecValTok{2}\NormalTok{, }\DecValTok{2}\NormalTok{, Row);}
\NormalTok{    a.transpose().print();}
    \CommentTok{// Or you can use shorter one}
\NormalTok{    a.t().print();}
    \CommentTok{//      c[0] c[1]}
    \CommentTok{// r[0]    1    3}
    \CommentTok{// r[1]    2    4}
\OperatorTok{\}}
\end{Highlighting}
\end{Shaded}
\end{itemize}

\hypertarget{lu-decomposition}{%
\section{LU Decomposition}\label{lu-decomposition}}

\begin{itemize}
\item
  Peroxide uses \textbf{complete pivoting} for LU decomposition - Very stable
\item
  Since there are lots of causes to generate error, you should use \texttt{Option}
\item
  \texttt{lu} returns \texttt{Option\textless{}PQLU\textgreater{}}

  \begin{itemize}
  \tightlist
  \item
    \texttt{PQLU} has four field - \texttt{p}, \texttt{q}, \texttt{l} , \texttt{u}
  \item
    \texttt{p} means row permutations
  \item
    \texttt{q} means column permutations
  \item
    \texttt{l} means lower triangular matrix
  \item
    \texttt{u} menas upper triangular matrix
  \end{itemize}
\item
  The structure of \texttt{PQLU} is as follows:

\begin{Shaded}
\begin{Highlighting}[]
\AttributeTok{#[}\NormalTok{derive}\AttributeTok{(}\BuiltInTok{Debug}\AttributeTok{,} \BuiltInTok{Clone}\AttributeTok{)]}
\KeywordTok{pub} \KeywordTok{struct}\NormalTok{ PQLU }\OperatorTok{\{}
    \KeywordTok{pub}\NormalTok{ p: Perms,}
    \KeywordTok{pub}\NormalTok{ q: Perms,}
    \KeywordTok{pub}\NormalTok{ l: Matrix,}
    \KeywordTok{pub}\NormalTok{ u: Matrix,}
\OperatorTok{\}}

\KeywordTok{pub} \KeywordTok{type}\NormalTok{ Perms = }\DataTypeTok{Vec}\NormalTok{<(}\DataTypeTok{usize}\NormalTok{, }\DataTypeTok{usize}\NormalTok{)>;}
\end{Highlighting}
\end{Shaded}
\item
  Example of LU decomposition:

\begin{Shaded}
\begin{Highlighting}[]
\KeywordTok{fn}\NormalTok{ main() }\OperatorTok{\{}
    \KeywordTok{let}\NormalTok{ a = matrix(}\PreprocessorTok{c!}\NormalTok{(}\DecValTok{1}\NormalTok{,}\DecValTok{2}\NormalTok{,}\DecValTok{3}\NormalTok{,}\DecValTok{4}\NormalTok{), }\DecValTok{2}\NormalTok{, }\DecValTok{2}\NormalTok{, Row);}
    \KeywordTok{let}\NormalTok{ pqlu = a.lu().unwrap(); }\CommentTok{// unwrap because of Option}
    \KeywordTok{let}\NormalTok{ (p,q,l,u) = (pqlu.p, pqlu.q, pqlu.l, pqlu.u);}
    \PreprocessorTok{assert_eq!}\NormalTok{(p, }\PreprocessorTok{vec!}\OperatorTok{[}\NormalTok{(}\DecValTok{0}\NormalTok{,}\DecValTok{1}\NormalTok{)}\OperatorTok{]}\NormalTok{); }\CommentTok{// swap 0 & 1 (Row)}
    \PreprocessorTok{assert_eq!}\NormalTok{(q, }\PreprocessorTok{vec!}\OperatorTok{[}\NormalTok{(}\DecValTok{0}\NormalTok{,}\DecValTok{1}\NormalTok{)}\OperatorTok{]}\NormalTok{); }\CommentTok{// swap 0 & 1 (Col)}
    \PreprocessorTok{assert_eq!}\NormalTok{(l, matrix(}\PreprocessorTok{c!}\NormalTok{(}\DecValTok{1}\NormalTok{,}\DecValTok{0}\NormalTok{,}\DecValTok{0.5}\NormalTok{,}\DecValTok{1}\NormalTok{),}\DecValTok{2}\NormalTok{,}\DecValTok{2}\NormalTok{,Row));}
    \CommentTok{//      c[0] c[1]}
    \CommentTok{// r[0]    1    0}
    \CommentTok{// r[1]  0.5    1}
    \PreprocessorTok{assert_eq!}\NormalTok{(u, matrix(}\PreprocessorTok{c!}\NormalTok{(}\DecValTok{4}\NormalTok{,}\DecValTok{3}\NormalTok{,}\DecValTok{0}\NormalTok{,-}\DecValTok{0.5}\NormalTok{),}\DecValTok{2}\NormalTok{,}\DecValTok{2}\NormalTok{,Row));}
    \CommentTok{//      c[0] c[1]}
    \CommentTok{// r[0]    4    3}
    \CommentTok{// r[1]    0 -0.5}
\OperatorTok{\}}
\end{Highlighting}
\end{Shaded}
\end{itemize}

\hypertarget{determinant}{%
\section{Determinant}\label{determinant}}

\begin{itemize}
\item
  Peroxide uses LU decomposition to obtain determinant (\(\mathcal{O}(n^3)\))

\begin{Shaded}
\begin{Highlighting}[]
\KeywordTok{fn}\NormalTok{ main() }\OperatorTok{\{}
    \KeywordTok{let}\NormalTok{ a = }\PreprocessorTok{matrix!}\NormalTok{(}\DecValTok{1}\NormalTok{;}\DecValTok{4}\NormalTok{;}\DecValTok{1}\NormalTok{, }\DecValTok{2}\NormalTok{, }\DecValTok{2}\NormalTok{, Row);}
    \PreprocessorTok{assert_eq!}\NormalTok{(a.det(), -}\DecValTok{2f64}\NormalTok{);}
\OperatorTok{\}}
\end{Highlighting}
\end{Shaded}
\end{itemize}

\hypertarget{inverse-matrix}{%
\section{Inverse matrix}\label{inverse-matrix}}

\begin{itemize}
\tightlist
\item
  Peroxide uses LU decomposition to obtain inverse matrix.
\item
  It needs two sub functions - \texttt{inv\_l}, \texttt{inv\_u}

  \begin{itemize}
  \tightlist
  \item
    For inverse of \texttt{L,\ U}, I use block partitioning. For example, for lower triangular matrix :
    \[\begin{aligned}
      L &= \begin{pmatrix}
      L_1 & \mathbf{0} \\
      L_2 & L_3 
      \end{pmatrix} \\
      L^{-1} &= \begin{pmatrix}
      L_1^{-1} & \mathbf{0} \\
      -L_3^{-1}L_2 L_1^{-1} & L_3^{-1}
      \end{pmatrix}
      \end{aligned}
      \]
  \end{itemize}

\begin{Shaded}
\begin{Highlighting}[]
\KeywordTok{fn}\NormalTok{ main() }\OperatorTok{\{}
    \KeywordTok{let}\NormalTok{ a = }\PreprocessorTok{matrix!}\NormalTok{(}\DecValTok{1}\NormalTok{;}\DecValTok{4}\NormalTok{;}\DecValTok{1}\NormalTok{, }\DecValTok{2}\NormalTok{, }\DecValTok{2}\NormalTok{, Row);}
\NormalTok{    a.inv().unwrap().print();}
    \CommentTok{//      c[0] c[1]}
    \CommentTok{// r[0]   -2    1}
    \CommentTok{// r[1]  1.5 -0.5}
\OperatorTok{\}}
\end{Highlighting}
\end{Shaded}
\end{itemize}

\hypertarget{moore-penrose-pseudo-inverse}{%
\section{Moore-Penrose Pseudo Inverse}\label{moore-penrose-pseudo-inverse}}

\begin{itemize}
\item
  \(X^\dagger = \left(X^T X\right)^{-1} X\)

\begin{Shaded}
\begin{Highlighting}[]
\KeywordTok{fn}\NormalTok{ main() }\OperatorTok{\{}
    \KeywordTok{let}\NormalTok{ a = }\PreprocessorTok{matrix!}\NormalTok{(}\DecValTok{1}\NormalTok{;}\DecValTok{4}\NormalTok{;}\DecValTok{1}\NormalTok{, }\DecValTok{2}\NormalTok{, }\DecValTok{2}\NormalTok{, Row);}
    \KeywordTok{let}\NormalTok{ pinv_a = a.psudo_inv().unwrap();}
    \KeywordTok{let}\NormalTok{ inv_a = a.inv().unwrap();}

    \PreprocessorTok{assert_eq!}\NormalTok{(inv_a, pinv_a); }\CommentTok{// Nearly equal (not actually equal)}
\OperatorTok{\}}

\CommentTok{// PartialEq implements}
\KeywordTok{impl} \BuiltInTok{PartialEq} \KeywordTok{for}\NormalTok{ Matrix }\OperatorTok{\{}
    \KeywordTok{fn}\NormalTok{ eq(&}\KeywordTok{self}\NormalTok{, other: &Matrix) -> }\DataTypeTok{bool} \OperatorTok{\{}
        \KeywordTok{if} \KeywordTok{self}\NormalTok{.shape == other.shape }\OperatorTok{\{}
            \KeywordTok{self}\NormalTok{.data.clone()}
\NormalTok{                .into_iter()}
\NormalTok{                .zip(other.data.clone())}
\NormalTok{                .all(|(x, y)| nearly_eq(x,y)) && }\KeywordTok{self}\NormalTok{.row == other.row}
        \OperatorTok{\}} \KeywordTok{else} \OperatorTok{\{}
            \KeywordTok{self}\NormalTok{.eq(&other.change_shape())}
        \OperatorTok{\}}
    \OperatorTok{\}}
\OperatorTok{\}}
\end{Highlighting}
\end{Shaded}
\end{itemize}

\hypertarget{functional}{%
\chapter{Functional Programming}\label{functional}}

\hypertarget{fp-for-vector}{%
\section{FP for Vector}\label{fp-for-vector}}

\begin{itemize}
\item
  There are some functional programming tools for \texttt{Vec\textless{}f64\textgreater{}}

\begin{Shaded}
\begin{Highlighting}[]
\KeywordTok{pub} \KeywordTok{trait}\NormalTok{ FPVector }\OperatorTok{\{}
    \KeywordTok{type}\NormalTok{ Scalar;}

    \KeywordTok{fn}\NormalTok{ fmap<F>(&}\KeywordTok{self}\NormalTok{, f: F) -> }\KeywordTok{Self}
    \KeywordTok{where}
\NormalTok{        F: }\BuiltInTok{Fn}\NormalTok{(}\KeywordTok{Self}\NormalTok{::Scalar) -> }\KeywordTok{Self}\NormalTok{::Scalar;}
    \KeywordTok{fn}\NormalTok{ reduce<F, T>(&}\KeywordTok{self}\NormalTok{, init: T, f: F) -> }\KeywordTok{Self}\NormalTok{::Scalar}
    \KeywordTok{where}
\NormalTok{        F: }\BuiltInTok{Fn}\NormalTok{(}\KeywordTok{Self}\NormalTok{::Scalar, }\KeywordTok{Self}\NormalTok{::Scalar) -> }\KeywordTok{Self}\NormalTok{::Scalar,}
\NormalTok{        T: }\PreprocessorTok{convert::}\BuiltInTok{Into}\NormalTok{<}\KeywordTok{Self}\NormalTok{::Scalar>;}
    \KeywordTok{fn}\NormalTok{ zip_with<F>(&}\KeywordTok{self}\NormalTok{, f: F, other: &}\KeywordTok{Self}\NormalTok{) -> }\KeywordTok{Self}
    \KeywordTok{where}
\NormalTok{        F: }\BuiltInTok{Fn}\NormalTok{(}\KeywordTok{Self}\NormalTok{::Scalar, }\KeywordTok{Self}\NormalTok{::Scalar) -> }\KeywordTok{Self}\NormalTok{::Scalar;}
    \KeywordTok{fn}\NormalTok{ filter<F>(&}\KeywordTok{self}\NormalTok{, f: F) -> }\KeywordTok{Self}
    \KeywordTok{where}
\NormalTok{        F: }\BuiltInTok{Fn}\NormalTok{(}\KeywordTok{Self}\NormalTok{::Scalar) -> }\DataTypeTok{bool}\NormalTok{;}
    \KeywordTok{fn}\NormalTok{ take(&}\KeywordTok{self}\NormalTok{, n: }\DataTypeTok{usize}\NormalTok{) -> }\KeywordTok{Self}\NormalTok{;}
    \KeywordTok{fn}\NormalTok{ skip(&}\KeywordTok{self}\NormalTok{, n: }\DataTypeTok{usize}\NormalTok{) -> }\KeywordTok{Self}\NormalTok{;}
\OperatorTok{\}}
\end{Highlighting}
\end{Shaded}
\end{itemize}

\hypertarget{fmap}{%
\subsection{fmap}\label{fmap}}

\begin{itemize}
\item
  \texttt{fmap} is syntactic sugar for \texttt{map}

\begin{Shaded}
\begin{Highlighting}[]
\KeywordTok{fn}\NormalTok{ main() }\OperatorTok{\{}
    \KeywordTok{let}\NormalTok{ a = }\PreprocessorTok{c!}\NormalTok{(}\DecValTok{1}\NormalTok{,}\DecValTok{2}\NormalTok{,}\DecValTok{3}\NormalTok{,}\DecValTok{4}\NormalTok{);}

    \CommentTok{// Original rust}
\NormalTok{    a.clone()}
\NormalTok{        .into_iter()}
\NormalTok{        .map(|x| x + }\DecValTok{1f64}\NormalTok{)}
\NormalTok{        .}\PreprocessorTok{collect::}\NormalTok{<}\DataTypeTok{Vec}\NormalTok{<}\DataTypeTok{f64}\NormalTok{>>()}
\NormalTok{        .print();}
        \CommentTok{// [2, 3, 4, 5]}

    \CommentTok{// fmap in Peroxide}
\NormalTok{    a.fmap(|x| x + }\DecValTok{1f64}\NormalTok{).print();}
    \CommentTok{// [2, 3, 4, 5]}
\OperatorTok{\}}
\end{Highlighting}
\end{Shaded}
\end{itemize}

\hypertarget{reduce}{%
\subsection{reduce}\label{reduce}}

\begin{itemize}
\item
  \texttt{reduce} is syntactic sugar for \texttt{fold}

\begin{Shaded}
\begin{Highlighting}[]
\KeywordTok{fn}\NormalTok{ main() }\OperatorTok{\{}
    \KeywordTok{let}\NormalTok{ a = }\PreprocessorTok{c!}\NormalTok{(}\DecValTok{1}\NormalTok{,}\DecValTok{2}\NormalTok{,}\DecValTok{3}\NormalTok{,}\DecValTok{4}\NormalTok{);}

    \CommentTok{// Original rust}
\NormalTok{    a.clone()}
\NormalTok{        .into_iter()}
\NormalTok{        .fold(}\DecValTok{0f64}\NormalTok{, |x, y| x + y)}
\NormalTok{        .print(); }\CommentTok{// 10}

    \CommentTok{// reduce in Peroxide}
\NormalTok{    a.reduce(}\DecValTok{0f64}\NormalTok{, |x, y| x + y).print(); }\CommentTok{// 10}
\OperatorTok{\}}
\end{Highlighting}
\end{Shaded}
\end{itemize}

\hypertarget{zip_with}{%
\subsection{zip\_with}\label{zip_with}}

\begin{itemize}
\item
  \texttt{zip\_with} is composed of \texttt{zip} \& \texttt{map}

\begin{Shaded}
\begin{Highlighting}[]
\KeywordTok{fn}\NormalTok{ main() }\OperatorTok{\{}
    \KeywordTok{let}\NormalTok{ a = }\PreprocessorTok{c!}\NormalTok{(}\DecValTok{1}\NormalTok{,}\DecValTok{2}\NormalTok{,}\DecValTok{3}\NormalTok{,}\DecValTok{4}\NormalTok{);}
    \KeywordTok{let}\NormalTok{ b = }\PreprocessorTok{c!}\NormalTok{(}\DecValTok{5}\NormalTok{,}\DecValTok{6}\NormalTok{,}\DecValTok{7}\NormalTok{,}\DecValTok{8}\NormalTok{);}

    \CommentTok{// Original rust}
\NormalTok{    a.clone()}
\NormalTok{        .into_iter()}
\NormalTok{        .zip(&b)}
\NormalTok{        .map(|(x, y)| x + *y)}
\NormalTok{        .}\PreprocessorTok{collect::}\NormalTok{<}\DataTypeTok{Vec}\NormalTok{<}\DataTypeTok{f64}\NormalTok{>>().print();}
        \CommentTok{// [6, 8, 10, 12]}

    \CommentTok{// zip_with in Peroxide}
\NormalTok{    a.zip_with(|x, y| x + y, &b).print();}
    \CommentTok{// [6, 8, 10, 12]}
\OperatorTok{\}}
\end{Highlighting}
\end{Shaded}
\end{itemize}

\hypertarget{filter}{%
\subsection{filter}\label{filter}}

\begin{itemize}
\item
  \texttt{filter} is just syntactic sugar for \texttt{filter}

\begin{Shaded}
\begin{Highlighting}[]
\KeywordTok{fn}\NormalTok{ main() }\OperatorTok{\{}
    \KeywordTok{let}\NormalTok{ a = }\PreprocessorTok{c!}\NormalTok{(}\DecValTok{1}\NormalTok{,}\DecValTok{2}\NormalTok{,}\DecValTok{3}\NormalTok{,}\DecValTok{4}\NormalTok{);}
\NormalTok{    a.filter(|x| x > }\DecValTok{2f64}\NormalTok{).print();}
    \CommentTok{// [3, 4]}
\OperatorTok{\}}
\end{Highlighting}
\end{Shaded}
\end{itemize}

\hypertarget{take-skip}{%
\subsection{take \& skip}\label{take-skip}}

\begin{itemize}
\item
  \texttt{take} is syntactic sugar for \texttt{take}

\begin{Shaded}
\begin{Highlighting}[]
\KeywordTok{fn}\NormalTok{ main() }\OperatorTok{\{}
    \KeywordTok{let}\NormalTok{ a = }\PreprocessorTok{c!}\NormalTok{(}\DecValTok{1}\NormalTok{,}\DecValTok{2}\NormalTok{,}\DecValTok{3}\NormalTok{,}\DecValTok{4}\NormalTok{);}
\NormalTok{    a.take(}\DecValTok{2}\NormalTok{).print();}
    \CommentTok{// [1, 2]}
\OperatorTok{\}}
\end{Highlighting}
\end{Shaded}
\item
  \texttt{skip} is syntactic sugar for \texttt{skip}

\begin{Shaded}
\begin{Highlighting}[]
\KeywordTok{fn}\NormalTok{ main() }\OperatorTok{\{}
    \KeywordTok{let}\NormalTok{ a = }\PreprocessorTok{c!}\NormalTok{(}\DecValTok{1}\NormalTok{,}\DecValTok{2}\NormalTok{,}\DecValTok{3}\NormalTok{,}\DecValTok{4}\NormalTok{);}
\NormalTok{    a.skip(}\DecValTok{2}\NormalTok{).print();}
    \CommentTok{// [3, 4]}
\OperatorTok{\}}
\end{Highlighting}
\end{Shaded}
\end{itemize}

\hypertarget{fp-for-matrix}{%
\section{FP for Matrix}\label{fp-for-matrix}}

\begin{itemize}
\item
  Similar to \texttt{FPVector}

\begin{Shaded}
\begin{Highlighting}[]
\KeywordTok{pub} \KeywordTok{trait}\NormalTok{ FP }\OperatorTok{\{}
    \KeywordTok{fn}\NormalTok{ take(&}\KeywordTok{self}\NormalTok{, n: }\DataTypeTok{usize}\NormalTok{, shape: Shape) -> Matrix;}
    \KeywordTok{fn}\NormalTok{ skip(&}\KeywordTok{self}\NormalTok{, n: }\DataTypeTok{usize}\NormalTok{, shape: Shape) -> Matrix;}
    \KeywordTok{fn}\NormalTok{ fmap<F>(&}\KeywordTok{self}\NormalTok{, f: F) -> Matrix }\KeywordTok{where}\NormalTok{ F: }\BuiltInTok{Fn}\NormalTok{(}\DataTypeTok{f64}\NormalTok{) -> }\DataTypeTok{f64}\NormalTok{;}
    \KeywordTok{fn}\NormalTok{ reduce<F, T>(&}\KeywordTok{self}\NormalTok{, init: T, f: F) -> }\DataTypeTok{f64} 
        \KeywordTok{where}\NormalTok{ F: }\BuiltInTok{Fn}\NormalTok{(}\DataTypeTok{f64}\NormalTok{, }\DataTypeTok{f64}\NormalTok{) -> }\DataTypeTok{f64}\NormalTok{,}
\NormalTok{            T: }\PreprocessorTok{convert::}\BuiltInTok{Into}\NormalTok{<}\DataTypeTok{f64}\NormalTok{>;}
    \KeywordTok{fn}\NormalTok{ zip_with<F>(&}\KeywordTok{self}\NormalTok{, f: F, other: &Matrix) -> Matrix }
        \KeywordTok{where}\NormalTok{ F: }\BuiltInTok{Fn}\NormalTok{(}\DataTypeTok{f64}\NormalTok{, }\DataTypeTok{f64}\NormalTok{) -> }\DataTypeTok{f64}\NormalTok{;}
\OperatorTok{\}}
\end{Highlighting}
\end{Shaded}
\item
  Above functions play same roles as \texttt{FPVector}
\end{itemize}

\hypertarget{statistics}{%
\chapter{Statistics}\label{statistics}}

\hypertarget{statistics-trait}{%
\section{\texorpdfstring{\texttt{Statistics} trait}{Statistics trait}}\label{statistics-trait}}

\begin{itemize}
\tightlist
\item
  To make generic code, there is \texttt{Statistics} trait

  \begin{itemize}
  \tightlist
  \item
    \texttt{mean}: just mean
  \item
    \texttt{var} : variance
  \item
    \texttt{sd} : standard deviation (R-like notation)
  \item
    \texttt{cov} : covariance
  \item
    \texttt{cor} : correlation coefficient
  \end{itemize}

\begin{Shaded}
\begin{Highlighting}[]
\KeywordTok{pub} \KeywordTok{trait}\NormalTok{ Statistics }\OperatorTok{\{}
    \KeywordTok{type}\NormalTok{ Array;}
    \KeywordTok{type}\NormalTok{ Value;}

    \KeywordTok{fn}\NormalTok{ mean(&}\KeywordTok{self}\NormalTok{) -> }\KeywordTok{Self}\NormalTok{::Value;}
    \KeywordTok{fn}\NormalTok{ var(&}\KeywordTok{self}\NormalTok{) -> }\KeywordTok{Self}\NormalTok{::Value;}
    \KeywordTok{fn}\NormalTok{ sd(&}\KeywordTok{self}\NormalTok{) -> }\KeywordTok{Self}\NormalTok{::Value;}
    \KeywordTok{fn}\NormalTok{ cov(&}\KeywordTok{self}\NormalTok{) -> }\KeywordTok{Self}\NormalTok{::Array;}
    \KeywordTok{fn}\NormalTok{ cor(&}\KeywordTok{self}\NormalTok{) -> }\KeywordTok{Self}\NormalTok{::Array;}
\OperatorTok{\}}
\end{Highlighting}
\end{Shaded}
\end{itemize}

\hypertarget{for-vecf64}{%
\subsection{\texorpdfstring{For \texttt{Vec\textless{}f64\textgreater{}}}{For Vec\textless{}f64\textgreater{}}}\label{for-vecf64}}

\begin{itemize}
\item
  Caution: For \texttt{Vec\textless{}f64\textgreater{}}, \texttt{cov} \& \texttt{cor} are unimplemented (those for \texttt{Matrix})

\begin{Shaded}
\begin{Highlighting}[]
\KeywordTok{fn}\NormalTok{ main() }\OperatorTok{\{}
    \KeywordTok{let}\NormalTok{ a = }\PreprocessorTok{c!}\NormalTok{(}\DecValTok{1}\NormalTok{,}\DecValTok{2}\NormalTok{,}\DecValTok{3}\NormalTok{,}\DecValTok{4}\NormalTok{,}\DecValTok{5}\NormalTok{);}
\NormalTok{    a.mean().print(); }\CommentTok{// 3}
\NormalTok{    a.var().print();  }\CommentTok{// 2.5}
\NormalTok{    a.sd().print();   }\CommentTok{// 1.5811388300841898}
\OperatorTok{\}}
\end{Highlighting}
\end{Shaded}
\item
  But there are other functions to calculate \texttt{cov} \& \texttt{cor}

\begin{Shaded}
\begin{Highlighting}[]
\KeywordTok{fn}\NormalTok{ main() }\OperatorTok{\{}
    \KeywordTok{let}\NormalTok{ v1 = }\PreprocessorTok{c!}\NormalTok{(}\DecValTok{1}\NormalTok{,}\DecValTok{2}\NormalTok{,}\DecValTok{3}\NormalTok{);}
    \KeywordTok{let}\NormalTok{ v2 = }\PreprocessorTok{c!}\NormalTok{(}\DecValTok{3}\NormalTok{,}\DecValTok{2}\NormalTok{,}\DecValTok{1}\NormalTok{);}

\NormalTok{    cov(&v1, &v2).print(); }\CommentTok{// -0.9999999999999998}
\NormalTok{    cor(&v1, &v2).print(); }\CommentTok{// -0.9999999999999993}
\OperatorTok{\}}
\end{Highlighting}
\end{Shaded}
\end{itemize}

\hypertarget{for-matrix}{%
\subsection{\texorpdfstring{For \texttt{Matrix}}{For Matrix}}\label{for-matrix}}

\begin{itemize}
\item
  For \texttt{Matrix}, \texttt{mean,\ var,\ sd} means column operations
\item
  \texttt{cov} means covariance matrix \& \texttt{cor} means also correlation coefficient matrix

\begin{Shaded}
\begin{Highlighting}[]
\KeywordTok{fn}\NormalTok{ main() }\OperatorTok{\{}
    \KeywordTok{let}\NormalTok{ m = matrix(}\PreprocessorTok{c!}\NormalTok{(}\DecValTok{1}\NormalTok{,}\DecValTok{2}\NormalTok{,}\DecValTok{3}\NormalTok{,}\DecValTok{3}\NormalTok{,}\DecValTok{2}\NormalTok{,}\DecValTok{1}\NormalTok{), }\DecValTok{3}\NormalTok{, }\DecValTok{2}\NormalTok{, Col);}

\NormalTok{    m.mean().print(); }\CommentTok{// [2, 2]}
\NormalTok{    m.var().print();  }\CommentTok{// [1.0000, 1.0000]}
\NormalTok{    m.sd().print();   }\CommentTok{// [1.0000, 1.0000]}

\NormalTok{    m.cov().print();}
    \CommentTok{//         c[0]    c[1]}
    \CommentTok{// r[0]  1.0000 -1.0000}
    \CommentTok{// r[1] -1.0000  1.0000}

\NormalTok{    m.cor().print();}
    \CommentTok{//         c[0]    c[1]}
    \CommentTok{// r[0]       1 -1.0000}
    \CommentTok{// r[1] -1.0000       1}
\OperatorTok{\}}
\end{Highlighting}
\end{Shaded}
\end{itemize}

\hypertarget{simple-random-number-generator}{%
\section{Simple Random Number Generator}\label{simple-random-number-generator}}

\begin{itemize}
\item
  Peroxide uses external \href{https://crates.io/crates/rand}{\texttt{rand} crate} to generate random number

\begin{Shaded}
\begin{Highlighting}[]
\KeywordTok{extern} \KeywordTok{crate}\NormalTok{ rand;}
\KeywordTok{use} \KeywordTok{self}\NormalTok{::}\PreprocessorTok{rand::prelude::}\NormalTok{*;}

\KeywordTok{fn}\NormalTok{ main() }\OperatorTok{\{}
    \KeywordTok{let} \KeywordTok{mut}\NormalTok{ rng = thread_rng();}

    \KeywordTok{let}\NormalTok{ a = rng.gen_range(}\DecValTok{0f64}\NormalTok{, }\DecValTok{1f64}\NormalTok{); }\CommentTok{// Generate random f64 number ranges from 0 to 1}
\OperatorTok{\}}
\end{Highlighting}
\end{Shaded}
\item
  To want more detailed explanation, see \href{https://crates.io/crates/rand}{\texttt{rand} crate}
\end{itemize}

\hypertarget{probability-distribution}{%
\section{Probability Distribution}\label{probability-distribution}}

\begin{itemize}
\tightlist
\item
  There are some famous pdf in Peroxide (not checked pdfs will be implemented soon)

  \begin{itemize}
  \tightlist
  \item[$\boxtimes$]
    Bernoulli
  \item[$\boxtimes$]
    Beta
  \item[$\square$]
    Dirichlet
  \item[$\boxtimes$]
    Gamma
  \item[$\boxtimes$]
    Normal
  \item[$\square$]
    Student's t
  \item[$\boxtimes$]
    Uniform
  \item[$\square$]
    Wishart
  \end{itemize}
\item
  There are two enums to represent probability distribution

  \begin{itemize}
  \tightlist
  \item
    \texttt{OPDist\textless{}T\textgreater{}} : One parameter distribution (Bernoulli)
  \item
    \texttt{TPDist\textless{}T\textgreater{}} : Two parameter distribution (Uniform, Normal, Beta, Gamma)

    \begin{itemize}
    \tightlist
    \item
      \texttt{T:\ PartialOrd\ +\ SampleUniform\ +\ Copy\ +\ Into\textless{}f64\textgreater{}}
    \end{itemize}
  \end{itemize}
\item
  There are some traits for pdf

  \begin{itemize}
  \tightlist
  \item
    \texttt{RNG} trait - extract sample \& calculate pdf
  \item
    \texttt{Statistics} trait - already shown above
  \end{itemize}
\end{itemize}

\hypertarget{rng-trait}{%
\subsection{\texorpdfstring{\texttt{RNG} trait}{RNG trait}}\label{rng-trait}}

\begin{itemize}
\tightlist
\item
  \texttt{RNG} trait is composed of two fields

  \begin{itemize}
  \tightlist
  \item
    \texttt{sample}: Extract samples
  \item
    \texttt{pdf} : Calculate pdf value at specific point
  \end{itemize}

\begin{Shaded}
\begin{Highlighting}[]
\KeywordTok{pub} \KeywordTok{trait}\NormalTok{ RNG }\OperatorTok{\{}
    \CommentTok{/// Extract samples of distributions}
    \KeywordTok{fn}\NormalTok{ sample(&}\KeywordTok{self}\NormalTok{, n: }\DataTypeTok{usize}\NormalTok{) -> }\DataTypeTok{Vec}\NormalTok{<}\DataTypeTok{f64}\NormalTok{>;}

    \CommentTok{/// Probability Distribution Function}
    \CommentTok{///}
    \CommentTok{/// # Type}
    \CommentTok{/// `f64 -> f64`}
    \KeywordTok{fn}\NormalTok{ pdf<S: }\BuiltInTok{PartialOrd}\NormalTok{ + SampleUniform + }\BuiltInTok{Copy}\NormalTok{ + }\BuiltInTok{Into}\NormalTok{<}\DataTypeTok{f64}\NormalTok{>>(&}\KeywordTok{self}\NormalTok{, x: S) -> }\DataTypeTok{f64}\NormalTok{;}
\OperatorTok{\}}
\end{Highlighting}
\end{Shaded}
\end{itemize}

\hypertarget{bernoulli-distribution}{%
\subsection{Bernoulli Distribution}\label{bernoulli-distribution}}

\hypertarget{uniform-distribution}{%
\subsection{Uniform Distribution}\label{uniform-distribution}}

\hypertarget{normal-distribution}{%
\subsection{Normal Distribution}\label{normal-distribution}}

\hypertarget{beta-distribution}{%
\subsection{Beta Distribution}\label{beta-distribution}}

\hypertarget{gamma-distribution}{%
\subsection{Gamma Distribution}\label{gamma-distribution}}

\bibliography{book.bib}


\end{document}
